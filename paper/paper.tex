% Documento LaTex com o artigo que estamos escrevendo

% Cabeçalho
% Onde a gente configura o documento
%%%%%%%%%%%%%%%%%%%%%%%%%%%%%%%%%%%%%%%%%%%%%%%%%%%%%
\documentclass{article}

% Deixa o pdf em português
\usepackage[brazil]{babel}
% Pacote para figura
\usepackage{graphicx}
\usepackage[round,authoryear,sort]{natbib}
\usepackage{notomath}

\newcommand{\Title}{Análise de variação de temperatura dos últimos 5 anos}

\input{paises.tex}

% \newcommand{\Paises}{\input{paises.tex}}

% Corpo
% Onde a gente escreve o texto
%%%%%%%%%%%%%%%%%%%%%%%%%%%%%%%%%%%%%%%%%%%%%%%%%%%%%%
\begin{document}

% Ambiente, entre begin e end

\title{\Title}
\author{Gilberto Dias, Yago Castro, Arthur Macêdo}

% Insere o título no pdf
\maketitle

\begin{abstract}
Meu resumo legalzão. \Title.
\end{abstract}

\section{Introdução}
Isso vai ser a minha introdução. 
Outra frase da introdução. 

Esse já será outro parágrafo da introdução.

Trabalho anteriores bem legal firezque faz coisas parecidas
\citep{Hansen2010}.
Isso foi analisado primeiro por \citet{Hansen2010}.

\section{Medotologia}
\label{sec:metodos}

Aqui eu vou descrever tudo que eu fiz.
Ajustamos uma reta aos cinco últimos anos dos dados
de temperatura média mensal para cada país.
Assim calculamos a taxa de variação de temperatura recente. 

A equação da reta é

%\begin{equation}
%y = \int_\Omega x dx,
%\end{equation}

\begin{equation}
T (t) = a t + b,
\label{eq:reta}
\end{equation}

\noindent
onde $T$ é a temperatura, $t$ é o tempo, $a$ é o coeficiente angular e 
$b$ é o coeficiente linear.

Utilizamos a equação \ref{eq:reta} em um código Python para fazer o ajuste da
reta com o método dos mínimos quadrados. Isso está descrito na seção \ref{sec:metodos}.

\section{Resultados}

Analisamos os dados de 225 países. 
Os países analisados foram: \Paises.

\begin{figure}[!htb]
	\centering
	\includegraphics[width=0.5\columnwidth]{../figuras/variacao_temperatura.png}
	\caption{
		Variação de temperatura média mensal dos cinco últimos anos.
		a) Países com as cinco maiores variação de temperatura.
		b) Países com as cinco menores variação de temperatura.
	}
	\label{fig:variacao}
\end{figure}

Os resultados da análise de variação de temperatura estão na figura \ref{fig:variacao}.

\bibliographystyle{apalike}
\bibliography{referencias.bib}

\end{document}